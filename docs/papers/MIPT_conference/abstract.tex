\documentclass[a4paper]{llncs}

\usepackage[utf8]{inputenc}
\usepackage[russian]{babel}
\renewcommand\andname{и}

\author{Зайченков П.О.\inst{1} \and Шинкаров А.Ю.\inst{2}}
\date{\today}
\title{Язык программирования Eq \\
      --- Аннотация ---}
\institute{
  Московский физико-технический институт,
  Кафедра информатики и вычислительной техники
\and
  University of Hertfordshire,
  Hatfield, Hertfordshire,
  AL10 9AB, United Kingdom
}

\begin{document}

\maketitle

Большинство естественнонаучных проблем сопряжено с
огромным количеством вычислений, которые варьируются от решения систем 
уравнений до обработки данных, полученных с 
телескопа.  Большинство подобных 
вычислений реализовано на языках программирования подобных Фортрану. На этот счет есть несколько причин.
Во-первых, обратная совместимость -- метеорологические программы насчитывают 
миллионы строк кода, а первые версии появились в тот момент, когда Фортран был одним из самых
передовых языков программирования; во-вторых, производительность --
языки высокого уровня, такие как MatLab, Python, Java предоставляют
высокий уровень абстракций, но не могут состязаться в скорости с
языками низкого уровня; и, наконец, сложность разработки -- Фортран
оказался компромиссом между скоростью разработки и скоростью
выполнения готовой программы. На сегодняшний день тенденции в производстве компьютеров сменились с 
гонки за тактовой частотой на увеличение количества ядер процессора.  
Именно этим фактом обусловлен рост интереса к параллельному программированию.  
Немаловажную роль играет появление на рынке графических ускорителей GPGPU,
предоставляющих еще б\'{о}льшие возможности для увеличения скорости
программ, однако требующие серьезного изменения парадигмы
программирования.

В данной статье мы представляем язык программирования Eq, который
позволит разделить усилия ученого, заинтересованного исключительно в
результате определенных вычислений и программиста, разрабатывающего
компилятор, соответствующий ситуации на современном рынке компьютеров.

Язык программирования для вычислительных задач должен обеспечивать высокую
производительность.  Ключевой момент при вычислениях --
разделение кода на два типа.  Первый тип -- независимые друг от друга блоки, которые могут выполняться параллельно.  Другой тип -- блоки, которые имеют зависимости, что означает, что инструкции
должны выполняться в строго определенной последовательности, чтобы не нарушить
зависимости между инструкциями.  Для того, чтобы выполнить некоторые участки
кода параллельно, необходимо либо указать на них явным образом, 
такой подход, к примеру, используется в 
библиотеках MPI и OpenMP, либо посредством некоторого анализа компилятор может 
выявить эти участки самостоятельно.  В случае Eq используется второй подход, 
однако, в качестве подсказки, в синтаксисе имеется две важные конструкции --
рекуррентное и параллельное выражение.  Рекуррентное выражение -- аналог нити 
исполнения (thread) в операционной системе, когда имеется некоторое окружение 
и последовательность действий, которая не может быть нарушена.  Параллельное 
выражение -- набор действий, который не имеет  зависимостей, и все действия 
могут быть выполнены одновременно.  Далее мы продемонстрируем, что двух данных 
конструкций, ветвления и атомарного выражения достаточно, чтобы, во-первых, 
записать любую программу на Фортране, во-вторых, принять решение о 
параллельности той или иной операции.

\end{document}