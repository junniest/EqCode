\documentclass[a4paper]{llncs}

\usepackage[utf8]{inputenc}
\usepackage[russian]{babel}
\renewcommand\andname{и}

\author{Зайченков П.О.\inst{1} \and Шинкаров А.Ю.\inst{2}}
\date{\today}
\title{Язык программирования Eq \\
      --- Аннотация ---}
\institute{
  Московский физико-технический институт,
  Кафедра информатики и вычислительной техники
\and
  University of Hertfordshire,
  Hatfield, Hertfordshire,
  AL10 9AB, United Kingdom
}

\begin{document}

\maketitle

Большинство современных естественнонаучных проблем тесно сопряжено с
огромным количеством вычислений.  Класс вычислений может
варьироваться от решения систем уравнений и моделирования физических процессов 
до обработки данных, полученных с телескопа, и
исследования организации геномов.  На данный момент, за редким
исключением, большинство подобных вычислений реализовано на языках
программирования подобных Фортрану.  На то существует несколько
причин: во-первых, обратная совместимость -- не секрет, что
метеорологические программы насчитывают миллионы строк кода, а первые
версии появились в тот момент, когда Фортран был одним из самых
передовых языков программирования; во-вторых, производительность --
языки высокого уровня, такие как MatLab, Python, Java предоставляют
высокий уровень абстракций, но редко могут состязаться в скорости с
языками низкого уровня; и, наконец, сложность разработки -- Фортран
оказался оптимальным компромиссом между скоростью разработки и скоростью
выполнения готовой программы -- более сложные оптимизации обычно
требуют от программиста нетривиальных знаний об архитектуре и
тонкостях того или иного языка.

Очевидно, что программы о которых идет речь выше, требуют огромной
вычислительной мощности.  На сегодняшний день тенденции в
производстве ЭВМ сменились с гонки за тактовой частотой на
увеличение количества ядер процессора.  Именно этим фактом обусловлен рост
интереса к параллельному программированию.  Немаловажную роль
играет появление на рынке графических ускорителей GPGPU,
предоставляющих еще б\'{о}льшие возможности для увеличения скорости
программ, однако требующие серьезного изменения парадигмы
программирования.

В данной статье мы представляем язык программирования Eq, который
позволит разделить усилия ученого, заинтересованного исключительно в
результате определенных вычислений и программиста, разрабатывающего
компилятор, соответствующий ситуации на современном рынке ЭВМ.
Основой синтаксиса для Eq является текстовый процессор \LaTeX,
являющийся стандартом для верстки научных публикаций.  Основное
преимущество выбранного подхода состоит в универсальности синтаксиса
Eq: с одной стороны программа понимается текстовым процессором и имеет
стандартный графический оттиск, с другой стороны, эта же программа
компилируется на большинстве современных архитектур.

Для того, чтобы выполнить некоторые участки кода параллельно, необходимо 
либо указать на них явным образом, такой подход, к примеру, используется в 
библиотеках MPI и OpenMP, либо посредством некоторого анализа компилятор может 
выявить эти участки самостоятельно.  В случае Eq используется второй подход, 
однако, в качестве подсказки, в синтаксисе имеется две важные конструкции --
рекуррентное и параллельное выражение.  Рекуррентное выражение -- аналог нити 
исполнения (thread) в операционной системе, когда имеется некоторое окружение 
и последовательность действий, которая не может быть нарушена.  Параллельное 
выражение -- набор действий, который не имеет  зависимостей, и все действия 
могут быть выполнены одновременно.  Далее мы продемонстрируем, что двух данных 
конструкций, ветвления и атомарного выражения достаточно, чтобы, во-первых, 
записать любую программу на Фортране, во-вторых, принять решение о 
параллельности той или иной операции.

\end{document}
